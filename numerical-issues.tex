\chapter{Numeric Artifacts}

%\setcounter{page}{1}

\renewcommand{\thefootnote}{\fnsymbol{footnote}}
\footnotetext{Lecture notes by John Schneider.  {\tt
numerical-issues.tex}}

\section{Introduction}

Virtually all solutions to problems in electromagnetics require the
use of a computer.  Even when an analytic or ``closed form'' solution
is available which is nominally exact, one typically must use a
computer to translate that solution into numeric values for a given
set of parameters.  Because of inherent limitations in the way numbers
are stored in computers, some errors will invariably be present in the
resulting solution.  These errors will typically be small but they are
an artifact about which one should be aware.  Here we will discuss
a few of the consequences of finite precision.

Later we will be discussing numeric solutions to electromagnetic
problems which are based on the finite-difference time-domain
(FDTD) method.  The FDTD method makes approximations that force
the solutions to be approximate, i.e., the method is inherently
approximate.  The results obtained from the FDTD method would be
approximate even if we used computers that offered infinite numeric
precision.  The inherent approximations in the FDTD method will be
discussed in subsequent chapters.

With numerical methods there is one note of caution which one should
always keep in mind.  Provided the implementation of a solution does
not fail catastrophically, a computer is always willing to give you a
result.  You will probably find there are times when, to get your
program simply to run, the debugging process is incredibly arduous.
When your program does run, the natural assumption is that all the
bugs have been fixed.  Unfortunately that often is not the case.
Getting the program to run is one thing, getting correct results is
another.  And, in fact, getting accurate results is yet another
thing---your solution may be correct for the given implementation, but
the implementation may not be one which is capable of producing
sufficiently accurate results.  Therefore, the more ways you have to
test your implementation and your solution, the better.  For example,
a solution may be obtained at one level of discretization and then
another solution using a finer discretization.  If the two solutions
are not sufficiently close, one has not yet converged to the ``true''
solution and a finer discretization must be used or, perhaps, there is
some systemic error in the implementation.  The bottom line: just
because a computer gives you an answer does not mean that answer is
correct.


\section{Finite Precision}

% Note: Originally use a program called one-third that summed 1/3
% three times.  But, that actually produces equal results!  Not sure
% how I got that.  Bummer.  Had that published for quite a while...
% Changed this to 1/11.
If we sum one-eleventh eleven times we know that the result is one,
i.e., $1/11+1/11+1/11+1/11+1/11+1/11+1/11+1/11+1/11+1/11=1$.  But is
that true on a computer?  Consider the C program shown in Program
\ref{pro:oneThird}.
\begin{program}
{\tt oneEleventh.c}: \index{oneThird.c@{\tt oneThird.c}}
Test if $1/11+1/11+1/11+1/11+1/11+1/11+1/11+1/11+1/11+1/11$ equals $1$. \label{pro:oneThird}
\codemiddle
\begin{lstlisting}
/* Is summing 1./11. ten times == 1.0? */
#include <stdio.h>

int main() {
  float a;

  a = 1.0 / 11.0;        /*@ \label{oneThirdA} @*/

  if (a + a + a + a + a + a + a + a + a + a + a == 1.0) /*@ \label{oneThirdB} @*/
    printf("Equal.\n");
  else
    printf("Not equal.\n");

  return 0;
}
\end{lstlisting}
\end{program}
In this program the float variable {\tt a} is set to one-eleventh.  In
line \ref{oneThirdB} the sum of eleven {\tt a}'s is compared to one.
If they are equal, the program prints ``Equal'' but prints ``Not
equal'' otherwise.  The output of this program is ``Not equal.''
Thus, to a computer (at least one running a language typically used in
the solution of electromagnetics problems),
the sum of one-eleventh eleven times is not equal to
one.  It is worth noting that had line \ref{oneThirdB} been written
{\tt a=1/11;}, {\tt a} would have been set to zero since integer math
would be used to evaluate the division.  By using {\tt a = 1.0 /
  11.0;}, the computer uses floating-point math.

The floating-point data types in C or FORTRAN can only store a finite
number of digits.  On most machines four bytes (32 binary digits or
bits) are used for single-precision numbers and eight bytes (64
digits) are used for double precision.  Returning to the sum of
one-elevenths, as an extreme example, assumed that a computer can only
store two decimal digits.  One eleventh is equal to 0.09090909\ldots
Thus, to two decimal places one-eleventh would be approximated by
0.09.  Summing this eleven times yields
\[
0.09 + 0.09 + 0.09 + 0.09 + 0.09 + 0.09 + 0.09 + 0.09 + 0.09 + 0.09 +
0.09 = 0.99
\]
which is clearly not equal to one.  If the number is stored with more
digits, the result becomes closer to one, but it never gets there.
Both the decimal and binary floating-point representation of
one-eleventh have an infinite number of digits.  Thus, when attempting
to store one-eleventh in a computer the number has to be truncated so
that the computer stores an approximation of one-eleventh.  Because of
this truncation summing one-eleventh eleven times does not yield one.

Since $1/10$ is equal to 0.1, it might appear this number can be stored
with a finite number of digits.  Although one-tenth has a finite
number of digits when written in base ten (decimal representation), it
has an infinite number of digits when written in base two (binary
representation).

In a floating-point decimal number each digit represents the number of
a particular power of ten.  Letting a blank represent a digit, 
a decimal number can be thought of in the follow way:\\
\begin{picture}(200,40)(-100,-15)
\put(0,7){\mbox{$\ldots$
          \rule{24pt}{1.5pt}\hspace{3pt}
          \rule{24pt}{1.5pt}\hspace{3pt}
          \rule{24pt}{1.5pt}\hspace{3pt}
          \rule{24pt}{1.5pt}\hspace{4pt} . \hspace{1pt}
          \rule{24pt}{1.5pt}\hspace{3pt}
          \rule{24pt}{1.5pt}\hspace{3pt}
          \rule{24pt}{1.5pt}\hspace{3pt}
          \rule{24pt}{1.5pt}\hspace{3pt}
          $\ldots$}}
\put(17,-10){
\mbox{$10^{3}$}\hspace{11pt}
\mbox{$10^{2}$}\hspace{11pt}
\mbox{$10^{1}$}\hspace{11pt}
\mbox{$10^{0}$}\hspace{19pt}
\mbox{$10^{-1}$}\hspace{3.9pt}
\mbox{$10^{-2}$}\hspace{3.9pt}
\mbox{$10^{-3}$}\hspace{3.9pt}
\mbox{$10^{-4}$}
}
\end{picture}\\
Each digits tells how many of a particular power of $10$ there is in a
number.  The decimal point serves as the dividing line between
negative and non-negative exponents.  Binary numbers are similar
except each digit represents a power or two:\\
\begin{picture}(200,40)(-100,-15)
\put(0,9){\mbox{$\ldots$
          \rule{24pt}{1.5pt}\hspace{3pt}
          \rule{24pt}{1.5pt}\hspace{3pt}
          \rule{24pt}{1.5pt}\hspace{3pt}
          \rule{24pt}{1.5pt}\hspace{4pt} . \hspace{1pt}
          \rule{24pt}{1.5pt}\hspace{3pt}
          \rule{24pt}{1.5pt}\hspace{3pt}
          \rule{24pt}{1.5pt}\hspace{3pt}
          \rule{24pt}{1.5pt}\hspace{3pt}
          $\ldots$}}
\put(20,-10){
\mbox{$2^{3}$}\hspace{16.5pt}
\mbox{$2^{2}$}\hspace{16.5pt}
\mbox{$2^{1}$}\hspace{16.5pt}
\mbox{$2^{0}$}\hspace{26pt}
\mbox{$2^{-1}$}\hspace{10pt}
\mbox{$2^{-2}$}\hspace{10pt}
\mbox{$2^{-3}$}\hspace{10pt}
\mbox{$2^{-4}$}
}
\end{picture}\\
The base-ten number $0.1_{10}$ is simply $1\times 10^{-1}$.  To
obtain the same value using binary numbers we have to take
$2^{-4}+2^{-5}+2^{-8}+2^{-9}+\ldots$, i.e., an infinite number of
binary digits.  Another way of writing this is
\[
0.1_{10} = 0.0001100110011001100110011\ldots_{2}.
\]
As before, when this is stored in a computer, the number has to be
truncated.  The stored value is no longer precisely equal to
one-tenth.  Summing ten of these values does not yield one (although
the difference is very small).

The details of how floating-point values are stored in a computer are
not a primary concern.  However, it is helpful to know how bits are
allocated.  Numbers are stored in exponential form and the standard
allocation of bits is:
\begin{center}
\begin{tabular}{lcccc}
                 & total bits & sign & mantissa & exponent \\
single precision &    32      &  1   &    23    &    8     \\
double precision &    64      &  1   &    52    &    11    
\end{tabular}
\end{center}
Essentially the exponent gives the magnitude of the number while the
mantissa gives the digits of the number---the mantissa determines the
precision.  The more digits available for the mantissa, the more
precisely a number can be represented.  Although a double-precision
number has twice as many total bits as a single-precision number, it
uses 52 bits for the mantissa whereas a single-precision number uses
23.  Therefore double-precision numbers actually offer more than twice
the precision of single-precision numbers.  A mantissa of 23 binary
digits corresponds to a little less than seven decimal digits.  This
is because $2^{23}$ is $8,\!388,\!608$, thus $23$ binary digits can
represent numbers between $0$ and $8,\!388,\!607$.  On the other hand,
a mantissa of $52$ binary digits corresponds to a value with between
$15$ and $16$ decimal digits
($2^{52}=4,\!503,\!599,\!627,\!370,\!496$).

For the exponent, a double-precision number has three more bits than a
single-precision number.  It may seem as if the double-precision
exponent has been short-changed as it does not have twice as many bits
as a single-precision number.  However, keep in mind that the exponent
represents the size of a number.  Each additional bit essentially
doubles the number of values that can be represented.  If the exponent
had nine bits, it could represent numbers which were twice as large as
single-precision numbers.  The three additional bits that a
double-precision number possesses allows it to represent exponents
which are eight times larger than single-precision numbers.  This
translates into numbers which are 256 times larger (or smaller) in
magnitude than single-precision numbers.

Consider the following equation
\[
 a+b=a.
\]
From mathematics we know this equation can only be satisfied if $b$ is
zero.  However, using computers this equation can be true, i.e., $b$
makes no contribution to $a$, even when $b$ is non-zero.

When numbers are added or subtracted, their mantissas are shifted
until their exponents are equal.  At that point the mantissas can be
directly added or subtracted.  However, if the difference in the
exponents is greater than the length of the mantissa, then the smaller
number will not have any affect when added to or subtracted from the
larger number.  The code fragment shown in Fragment \ref{pro:bigSmall}
illustrates this phenomenon.
\begin{fragment}
Code fragment to test if a non-zero $b$ can satisfy the equation
$a+b=a$. \label{pro:bigSmall}
\codemiddle
\begin{lstlisting}
  float a = 1.0, b = 0.5, c;

  c = a + b;

  while(c != a) { /*@ \label{fragCompareA} @*/
    b = b / 2.0;
    c = a + b;
  }

  printf("%12g %12g %12g\n",a,b,c); /*@ \label{fragCompareB} @*/
\end{lstlisting}
\end{fragment}
Here $a$ is initialized to one while $b$ is set to one-half.  The
variable $c$ holds the sum of $a$ and $b$.  The while-loop starting on
line \ref{fragCompareA} will continue as long as $c$ is not equal to
$a$.  In the body of the loop, $b$ is divided by $2$ and $c$ is again
set equal to $a+b$.  If the computer had infinite precision, this
would be an infinite loop.  The value of $b$ would become vanishingly
small, but it would never be zero and hence $a+b$ would never equal
$a$.  However, the loop does terminate and the output of the {\tt
printf()} statement in line \ref{fragCompareB} is:
\begin{code}
           1   5.96046e-08   1
\end{code}
This shows that both $a$ and $c$ are unity while $b$ has a value of 
$5.96046\times10^{-8}$.  Note that this value of $b$ corresponds to
$1\times 2^{-24}$.  When $b$ has this value, the exponents of $a$ and $b$
differ by more than $23$ ($a$ is $1\times 2^0$).

One more example serves to illustrate the less-than-obvious ways in
which finite precision can corrupt a calculation.  Assume the variable
$a$ is set equal to $2$.  Taking the square root of $a$ and then
squaring $a$ should yield a result which is close to $2$ (ideally it
would be $2$, but since $\sqrt{2}$ has an infinite number of digits,
some accuracy will be lost).  However, what happens if the square root
is taken 23 times and then the number is squared 23 times?  We would
hope to get a result close to two, but that is not the case.
The program shown in Program \ref{pro:squareRoot} allows us to test
this scenario.
\begin{program}
{\tt rootTest.c}: \index{rootTest.c@{\tt rootTest.c}}
Take the square root of a number repeatedly
and then squaring the number an equal number of
times. \label{pro:squareRoot}
\codemiddle
\begin{lstlisting}
/* Square-root test. */
#include <math.h>  // needed for sqrt()
#include <stdio.h>

#define COUNT 23

int main() {
  float a = 2.0;
  int i;

  for (i = 0; i < COUNT; i++)
    a = sqrt(a);  // square root of a

  for (i = 0; i < COUNT; i++)
    a = a * a;      // a squared

  printf("%12g\n",a);

  return 0;
}
\end{lstlisting}
\end{program}
The program output is one, i.e., the result is $a=1.0$.  Each time the
square root is taken, the value gets closer and closer to unity.
Eventually, because of truncation error, the computer thinks the
number is unity.  At that point no amount of squaring the number will
change it.

\section{Symbolic Manipulation}

When using languages which are typically used in numerical analysis
(such as C, C++, FORTRAN, or even Matlab), truncation error is
unavoidable.  The ratio of the circumference of a circle to its
diameter is the number $\pi=3.141592\ldots$\@ This is an irrational
number with an infinite number of digits.  Thus one cannot store the
exact numeric value of $\pi$ in a computer.  Instead, one must use an
approximation consisting of a finite number of digits.  However, there
are software packages, such a Mathematica, that allow one to
manipulate symbols.  Within Mathematica, if a person writes {\tt Pi},
Mathematica ``knows'' symbolically what that means.  For example, the
cosine of {\tt 10000000001*Pi} is identically negative one.
Similarly, one could write {\tt Sqrt[2]}.  Mathematica knows that the
square of this is identically 2.  Unfortunately, though, such symbolic
manipulations are incredibly expensive in terms of computational
resources.  Many cutting-edge problems in electromagnetics can involve
hundreds of thousand or even millions of unknowns.  To deal with these
large amounts of data it is imperative to be as efficient---both in
terms of memory and computation time---as possible.  Mathematica is
wonderful for many things, but it is not the right tool for solving
large numeric problems.

In Matlab one can write {\tt pi} as a shorthand representation of
$\pi$.  However, this representation of $\pi$ is different from that
used in Mathematica.  In Matlab, {\tt pi} is essentially the same as
the numeric representation---it is just more convenient to write {\tt
pi} than all the numeric digits.  In C, provided you have included the
header file {\tt math.h}, you can use {\tt M\_PI} as a shorthand for
$\pi$.  Looking in {\tt math.h} reveals the following statement:
\begin{verbatim}
# define M_PI           3.14159265358979323846  /* pi */
\end{verbatim}
This is similar to what is happening in Matlab.  Matlab only knows
what the numeric value of {\tt pi} is and that numeric value is a
truncated version of the true value.  Thus, taking the cosine of {\tt
10000000001*pi} yields $-0.99999999999954$ instead of the exact value
of $-1$ (but, of course, the difference is trivial in this case).

