\chapter{Construction of Fourth-Order Central Differences \label{ap:diff}}

%\setcounter{page}{1}

% now do this kind of numbering everywhere
%\renewcommand{\thepage}{\Alph{chapter}.\arabic{page}}

Assuming a uniform spacing of $\delta$ between sample points, we seek
an approximation of the derivative of a function at $x_0$ which falls
midway between two sample point.  Taking the Taylor series expansion
of the function at the four sample points nearest to $x_0$ yields
\begin{eqnarray}
  f\!\left(x_0+\frac{3\delta}{2}\right) &=&
    f(x_0) + \frac{3\delta}{2} f'(x_0) +
        \frac{1}{2!}\left(\frac{3\delta}{2}\right)^2 f''(x_0) + 
        \frac{1}{3!}\left(\frac{3\delta}{2}\right)^3 f'''(x_0) + \ldots,
	\label{eq:fourI}
  \\
  f\!\left(x_0+\frac{\delta}{2}\right) &=&
    f(x_0) + \frac{\delta}{2} f'(x_0) +
        \frac{1}{2!}\left(\frac{\delta}{2}\right)^2 f''(x_0) + 
        \frac{1}{3!}\left(\frac{\delta}{2}\right)^3 f'''(x_0) + \ldots,
	\label{eq:fourII}
  \\
  f\!\left(x_0-\frac{\delta}{2}\right) &=&
    f(x_0) - \frac{\delta}{2} f'(x_0) +
        \frac{1}{2!}\left(\frac{\delta}{2}\right)^2 f''(x_0) -
        \frac{1}{3!}\left(\frac{\delta}{2}\right)^3 f'''(x_0) + \ldots,
	\label{eq:fourIII}
  \\
  f\!\left(x_0-\frac{3\delta}{2}\right) &=&
    f(x_0) - \frac{3\delta}{2} f'(x_0) +
        \frac{1}{2!}\left(\frac{3\delta}{2}\right)^2 f''(x_0) -
        \frac{1}{3!}\left(\frac{3\delta}{2}\right)^3 f'''(x_0) + \ldots
	\label{eq:fourIV}
\end{eqnarray}
Subtracting \refeq{eq:fourIII} from \refeq{eq:fourII} and 
\refeq{eq:fourIV} from \refeq{eq:fourI} yields
\begin{eqnarray}
  f\!\left(x_0+\frac{\delta}{2}\right) -
  f\!\left(x_0-\frac{\delta}{2}\right) &=& 
   \delta f'(x_0) +
        \frac{2}{3!}\left(\frac{\delta}{2}\right)^3 f'''(x_0) + \ldots
	\label{eq:inner}
  \\
  f\!\left(x_0+\frac{3\delta}{2}\right) -
  f\!\left(x_0-\frac{3\delta}{2}\right) &=& 
   3 \delta f'(x_0) +
        \frac{2}{3!}\left(\frac{3\delta}{2}\right)^3 f'''(x_0) + \ldots
	\label{eq:outer}
\end{eqnarray}
The goal now is to eliminate the term containing $f'''(x_0)$.  This
can be accomplished by multiplying \refeq{eq:inner} by 27 and then
subtracting \refeq{eq:outer}.  The result is
\begin{equation}
  27 f\!\left(x_0+\frac{\delta}{2}\right) -
  27 f\!\left(x_0-\frac{\delta}{2}\right) 
  - f\!\left(x_0+\frac{3\delta}{2}\right) +
  f\!\left(x_0-\frac{3\delta}{2}\right) =
    24 \delta f'(x_0) + O(\delta^5).
\end{equation}
Solving for $f'(x_0)$ yields
\begin{equation}
  \left.\frac{df(x)}{dx}\right|_{x=x_0} = \frac{9}{8} 
  \frac{f\!\left(x_0+\frac{\delta}{2}\right) -
        f\!\left(x_0-\frac{\delta}{2}\right)}{\delta}
  -\frac{1}{24}
  \frac{f\!\left(x_0+\frac{3\delta}{2}\right) -
        f\!\left(x_0-\frac{3\delta}{2}\right)}{\delta}
    + O(\delta^4).
\end{equation}
The first term on the right-hand side is the contribution from the
sample points nearest $x_0$ and the second term is the contribution
from the next nearest points.  The highest-order term not shown is
fourth-order in terms of $\delta$.  

%%%%%%%%%%%%%%%%%%%%%%%%%%%%%%%%%%%%%%%%%%%%%%%%%%%%%%%%%%%%%%%%%%%%%%%%%
\chapter{Generating a Waterfall Plot and Animation
\label{ap:waterfall}} 

%\setcounter{page}{1}
\index{waterfall plot|(}

Assume we are interested in plotting multiple snapshots of
one-dimensional data.  A waterfall plot displays all the snapshots in
a single image where each snapshot is offset slightly from the next.
On the other hand, animations display one image at a time, but
cycle through the images quickly enough so that one can clearly
visualize the temporal behavior of the field.  Animations are a
wonderful way to ascertain what is happening in an FDTD simulation
but, since there is no way to put an animation on a piece of paper,
waterfall plots also have a great deal of utility.  

We begin by discussing waterfall plots.  First, the data from the
individual frames must be loaded into Matlab.  The m-file for a
function to accomplish this is shown in Program
\ref{pro:readOneD}.  This function, {\tt readOneD()}, reads each frame
and stores the data into a matrix---each row of which corresponds to
one frame.  {\tt readOneD()} takes a single argument corresponding to
the base name of the frames.  For example, issuing the command {\tt z
= readOneD('sim');} would create a matrix {\tt z} where the first row
corresponded to the data in file {\tt sim.0}, the second row to the
data in file {\tt sim.1}, and so on.

Note that there is a {\tt waterfall()} function built into Matlab.
One could use that to display the data by issuing the command {\tt
waterfall(z)}.  However, the built-in command is arguably overkill.
Its hidden-line removal and colorization slow the rendering and do not
necessarily aide in the visualization of the data.

The plot shown in Fig.\ \ref{fig:waterfall} did not use Matlab's
{\tt waterfall()} function.  Instead, it was generated using a
function called {\tt simpleWaterfall()} whose m-file is shown in
Program \ref{pro:waterfall}.  This command takes three arguments.  The
first is the matrix which would be created by {\tt readOneD()}, the
second is the vertical offset between successive plots, and the third
is a vertical scale factor.

Given these two m-files, Fig.\ \ref{fig:waterfall} was generated using
the following commands:
\small
\renewcommand{\baselinestretch}{1.0}
\begin{verbatim}
z = readOneD('sim');
simpleWaterfall(z, 1, 1.9)  % vertical offset = 1, scale factor = 1.9
xlabel('Space [spatial index]')
ylabel('Time [frame number]')
\end{verbatim}
\normalsize
\renewcommand{\baselinestretch}{1.0}

\begin{program}
{\tt readOneD.m} Matlab code to read one-dimensional data from a
series of frames. \label{pro:readOneD} 
\codemiddle
\begin{lstlisting}[language=Matlab]
function z = readOneD(basename)
%readOneD(BASENAME) Read 1D data from a series of frames.
%  [Z, dataLength, nFrames] = readOneD(BASENAME) Data
%  is read from a series of data files all which have
%  the common base name given by the string BASENAME,
%  then a dot, then a frame index (generally starting
%  with zero).  Each frame corresponds to one row of Z.

% read the first frame and establish length of data
nFrames = 0;
filename = sprintf('%s.%d', basename, nFrames);
nFrames = nFrames + 1;
if exist(filename, 'file')
  z = dlmread(filename, '\n');
  dataLength = length(z);
else
  return;
end

% loop through other frames and break out of loop
% when next frame does not exist
while 1
  filename = sprintf('%s.%d',basename,nFrames);
  nFrames = nFrames + 1;
  if exist(filename, 'file')
    zTmp = dlmread(filename, '\n');
    if length(zTmp) ~= dataLength  % check length matches
      error('Frames have different sizes.')
      break;
    end
    z = [z zTmp]; % append new data to z
  else
    break;
  end
end

% reshape z to appropriate dimensions
z = reshape(z, dataLength, nFrames - 1);
z = z';

return;
\end{lstlisting}
\end{program}

\begin{program}
{\tt simpleWaterfall.m} Matlab function to generate a simple waterfall
plot. \label{pro:waterfall} 
\codemiddle
\begin{lstlisting}[language=Matlab]
function simpleWaterfall(z, offset, scale)
%simpleWaterfall Waterfall plot from offset x-y plots.
%  simpleWaterfall(Z, OFFSET, SCALE) Plots each row of z
%  where successive plots are offset from each other by
%  OFFSET and each plot is scaled vertically by SCALE.

hold off               % release any previous plot
plot(scale * z(1, :))  % plot the first row
hold on                % hold the plot

for i = 2:size(z, 1)   % plot the remaining rows
  plot(scale * z(i, :) + offset * (i - 1))
end

hold off               % release the plot

return
\end{lstlisting}
\end{program}

A function to generate an animation of one-dimensional data sets is
shown in Program \ref{pro:animateOneD}.  There are multiple ways to
accomplish this goal and thus one should keep in mind that Program
\ref{pro:animateOneD} is not necessarily the best approach for a
particular situation.  The function in Program \ref{pro:animateOneD}
is called {\tt oneDmovie()} and it takes three arguments: the base
name of the snapshots, and the minimum and maximum values of the
vertical axis.  The function uses a loop, starting in line $25$, to
read each of the snapshots.  Each snapshot is plotted and the plot
recorded as a frame of a Matlab ``movie'' (see the Matlab command {\tt
movie()} for further details).  The {\tt oneDmovie()} function returns
an array of movie frames.  As an example, assume the base name is
``{\tt sim}'' and the user wants the plots to range from $-1.0$ to
$1.0$.  The following commands would display the animation $10$ times
(the second argument to the {\tt movie()} command controls how often
the movie is repeated):
\small
\renewcommand{\baselinestretch}{1.0}
\begin{verbatim}
reel = oneDmovie('sim',-1,1);
movie(reel,10)
\end{verbatim}
\normalsize
\renewcommand{\baselinestretch}{1.0}
An alternative implementation might read all the data first, as was
done with the waterfall plot, and then determine the ``global''
minimum and maximum values from the data itself.  This would free the
user from specify those value as {\tt oneDmovie()} currently requires.
Such an implementation is left to the interested reader.


\begin{program}
{\tt oneDmovie.m} Matlab function which can be used to generate an
animation for multiple one-dimensional data sets.  For further
information on Matlab movies, see the Matlab command {\tt
movie}. \label{pro:animateOneD} 
\codemiddle
\begin{lstlisting}[language=Matlab]
function reel = oneDmovie(basename, y_min, y_max)
% oneDmovie Create a movie from data file with a common base
%    name which contain 1D data.
%
% basename = common base name of all files
% y_min    = minimum value used for all frames
% y_max    = maximum value used for all frames
%
% reel = movie which can be played with a command such as:
%          movie(reel, 10)
%    This would play the movie 10 times.  To control the frame
%    rate, add a third argument specifying the desired rate.

% open the first frame (i.e., first data file).
frame = 1;
filename = sprintf('%s.%d', basename, frame);
fid = fopen(filename, 'rt');

% to work around rendering bug under Mac OS X see:
% <www.mathworks.com/support/solutions/
%  data/1-VW0GM.html?solution=1-VW0GM>
figure; set(gcf, 'Renderer', 'zbuffer');

% provided fid is not -1, there is another file to process
while fid ~= -1
  data=fscanf(fid, '%f');    % read the data
  plot(data)                % plot the data
  axis([0 length(data) y_min y_max]) % scale axes appropriately
  reel(frame) = getframe;   % capture the frame for the movie

  % construct the next file name and try to open it
  frame = frame + 1;
  filename = sprintf('%s.%d', basename, frame);
  fid = fopen(filename, 'rb');
end

return
\end{lstlisting}
\end{program}

\index{waterfall plot|)}



%%%%%%%%%%%%%%%%%%%%%%%%%%%%%%%%%%%%%%%%%%%%%%%%%%%%%%%%%%%%%%%%%%%%%%%%%
\chapter{Rendering and Animating Two-Dimensional Data
\label{ap:render2d}} 

%\setcounter{page}{1}

The function shown below is Matlab code that can be used to generate a
movie from a sequence of binary (raw) files.  The files (or frames)
are assumed to be named such that they share a common base name then
have a dot followed by a frame number.  Here the frame number is
assumed to start at zero.  The function can have one, two, or three
arguments.  This first argument is the base name, the second is the
value which is used to normalize all the data, and the third argument
specifies the number of decades of data to display.  Here the absolute
value of the data is plotted in a color-mapped imaged.  Logarithmic
(base 10) scaling is used so that the value which is normalized to
unity will correspond to zero on the color scale and the smallest
normalized value will correspond, on the color scale, to the negative
of the number of decades (e.g., if the number of decades were three,
the smallest value would correspond to $-3$).  This smallest
normalized value actually corresponds to a normalized value of
$10^{-d}$ where $d$ is the number of decades.  Thus the (normalized)
values shown in the output varying from $10^{-d}$ to $1$.  The default
normalization and number of decades are $1$ and $3$, respectively.

\begin{program}
{\tt raw2movie.m} Matlab function to generate a movie given a sequence
of raw files.
\label{pro:raw2movie}
\codemiddle
\begin{lstlisting}[language=Matlab]
function reel = raw2movie(basename, z_norm, decades)
% raw2movie Creates a movie from "raw" files with a common base
%      name.
%
% The absolute value of the data is used together with
% logarithmic scaling.  The user may specify one, two, or
% three arguments.
% raw2movie(basename, z_norm, decades) or
% raw2movie(basename, z_norm) or
% raw2movie(basename):
% basename = common base name for all files
% z_norm   = value used to normalize all frames, typically this
%         would be the maximum value for all the frames.
%         Default value is 1.
% decades  = decades to be used in the display.  The normalized
%         data is assumed to vary between 1.0 and 10^(-decades)
%         so that after taking the log (base 10), the values
%         vary between 0 and -decades.  Default value is 3.
%
% return value:
% reel = movie which can be played with a command such as:
%          movie(reel, 10)
%        pcolor() is used to generate the frames.
%
% raw file format:
% The raw files are assumed to consist of all floats (in
% binary format).  The first two elements specify the horizontal
% and vertical dimensions.  Then the data itself is given in
% English book-reading order, i.e., from the upper left corner
% of the image and then scanned left to right.  The frame number
% is assumed to start at zero.

% set defaults if we have less than three arguments
if nargin < 3, decades = 3; end
if nargin < 2, z_norm = 1.0; end

% open the first frame
frame = 0;
filename = sprintf('%s.%d', basename, frame);
fid = fopen(filename, 'rb');

if fid == -1
  error(['raw2movie: initial frame not found: ', filename])
end

% to work around rendering bug under Mac OS X implementation.
figure; set(gcf, 'Renderer', 'zbuffer');

% provided fid is not -1, there is another file to process
while fid ~= -1
  size_x = fread(fid, 1, 'single');
  size_y = fread(fid, 1, 'single');

  data = flipud(transpose(...
           reshape(...
             fread(fid, size_x * size_y, 'single'), size_x, size_y)...
         ));

  % plot the data
  if decades ~= 0
    pcolor(log10(abs((data + realmin) / z_norm)))
    shading flat
    axis equal
    axis([1 size_x 1 size_y])
    caxis([-decades 0])
    colorbar
  else
    pcolor(abs((data + realmin) / z_norm))
    shading flat
    axis equal
    axis([1 size_x 1 size_y])
    caxis([0 1])
    colorbar
  end

  % capture the frame for the movie (Matlab wants index to start
  % at 1, not zero, hence the addition of one to the frame)
  reel(frame + 1) = getframe;

  % construct the next file name and try to open it
  frame = frame + 1;
  filename = sprintf('%s.%d', basename, frame);
  fid = fopen(filename,' rb');

end
\end{lstlisting}
\end{program}


%%%%%%%%%%%%%%%%%%%%%%%%%%%%%%%%%%%%%%%%%%%%%%%%%%%%%%%%%%%%%%%%%%%%%%%%%
\chapter{Notation}

%\setcounter{page}{1}

\begin{tabular}{ll}
$c$ & speed of light {\em in free space}\\
$N_{\mbox{\scriptsize freq}}$ &
        index of spectral component corresponding to a frequency
        with discretization $\ppw$ \\
$\ppw$ & 
        number of points per wavelength for a given frequency \\
      & (the wavelength is the one pertaining to propagation in free
        space) \\
$N_L$ & number of points per skin depth ($L$ for loss)\\
$N_P$ & number of points per wavelength at peak frequency of a Ricker
        wavelet  \\
$N_T$ & number of time steps in a simulation \\
$S_c$ & Courant number ($c\Delt/\Delx$ in one dimension)\\
$s_t$ & Temporal shift operator \\
$s_x$ & Spatial shift operator (in the $x$ direction) \\
\end{tabular}
