\documentclass[12pt]{report}

\topmargin 0pt
\headheight 0pt
\headsep 0pt
\textwidth 6.5in
\textheight 9.0in
\oddsidemargin 0in
\evensidemargin 0in
\raggedbottom

\usepackage{times}        % For making nicer PDF files:
\usepackage{amsmath}      % Good math stuff
\usepackage{epsfig}

\usepackage{float}
\floatstyle{ruled}
\newfloat{Program}{thp}{pro}

%\parskip 12pt
%\parindent 0pt

\begin{document}
\begin{center}
{\bf\large
EE 535 Lecture Notes Policy
}
\end{center}

I am willing to try to give copies of my lecture notes that will
provide a narrative similar to the one presented in class.  These
notes should not be considered a substitute for the textbook.  Owing
to time and software constraints I may not be able to include figures
or other material (although the notes will probably include space
where the figures should go).  Additionally, there may be times when
other things conspire to keep me from completing a set of notes and
hence there may be gaps between the prepared lecture notes and the
lectures themselves.

The provision of lecture notes should not be viewed as a reason not to
take notes in class---especially since figures may be missing from
the supplied notes.  I encourage you to takes note during the lecture.

To help motivate the reading of lecture notes and to help improve
their quality, I will award bonus points if you catch errors in the
notes.  The value of the bonuses will be as follows:

\begin{itemize}
\item Simple typographical errors in the narrative are worth 0.1 points.
Examples of such errors are misspellings, erroneously repeated words,
or grammatic errors (like improper subject-verb agreement).
\item Typographical errors in an equation are worth 0.2 points.  
\item Technical errors are worth 0.5 points.  For example, if I
claimed that ``forward differences'' have second-order accuracy, you
would be awarded 0.5 bonus points if you pointed out that they
actually have first-order accuracy.
\item Identification of ``missing material'' is worth 0.25 points.
When describing a topic with which you are quite familiar to those who
are unfamiliar with it, it is all too easy to omit material which is
important to the understanding of that topic.  If you are confused by
the discussion of some subject it may be because I left out something
which is key to understanding that subject.  On the other hand, it may
be because you did not recall something that you really should have
recalled (or perhaps you simply never were taught something that you
really should have been taught by this point in your education).  If
you can convince me that the narrative would really be better with
some additional material, that is worth 0.25 points.
\end{itemize}

A tenth of a point may not sound like much, but, unfortunately for me,
I tend to make a lot of typographical errors.  Therefore, bonus points
will only be awarded to the first person to identify the error.  You
may acquire no more than four points total.  I find that sometimes
sentences completely fall apart when they are initially written.
Sometimes those mangled sentences sneak through proof-reading.
Therefore you cannot earn multiple points for multiple errors in a
single sentence.  The bonus will be applied after calculating final
grades in the absence of a bonus (a four-point bonus could make a
difference of a letter grade).  Thus the bonus will not hurt those who
do not have bonus points---a bonus will only help those who do.

Lecture notes will be available at the class Web site and will be
updated as errors are corrected (I will not redistribute hard-copies
unless a change is quite significant).

\end{document}
